\input{header.tex}
\title{}
\author{}
\begin{document}
\maketitle

\end{document}

Sets font size to 11pt
Includes commonly needed packages
Reduces page margins to 0.75in for more space
Redefines \maketitle to save space
Removes page numbers
Changes enumeration to letters as in physics exercises
\v{ } makes bold vectors (\v is redefined to \vaccent)
\uv{ } makes bold unit vectors with hats
\gv{ } makes bold vectors of greek letters
\abs{ } makes the absolute value symbol
\avg{ } makes the angled average symbol
\d{ }{ } makes derivatives (\d is redefined to \underdot)
\dd{ }{ } makes double derivatives
\pd{ }{ } makes partial derivatives
\pdd{ }{ } makes double partial derivatives
\pdc{ }{ }{ } makes thermodynamics partial derivatives
\ket{ } makes Dirac kets
\bra{ } makes Dirac bras
\braket{ }{ } makes Dirac brackets
\matrixel{ }{ }{ } makes Dirac matrix elements
\grad{ } makes a gradient operator
\div{ } makes a divergence operator (\div is redefined to \divsymb)
\curl{ } makes a curl operator
\={ } makes numbers appear over equal signs (\= is redefined to \baraccent) 

General LaTeX tips:

Use "$ ... $" for inline equations
Use "\[ ... \]" for equations on their own line
Use "\begin{center} ... \end{center}" to center something
Use "\includegraphics[width=?cm]{filename.eps}" for images - must compile to dvi then use dvipdfm from a batch file
Use "\begin{multicols}{2} ... \end{multicols}" for two columns
Use "\begin{enumerate} \item ... \end{enumerate}" for parts of physics exercises
Use "\section*{ }" for sections without numbering
Use "\begin{cases} ... \end{cases}" for piecewise functions
Use "\mathcal{ }" for a caligraphic font
Use "\mathbb{ }" for a blackboard bold font 
