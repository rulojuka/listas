\subsection*{\underline{Dependências}:}

O Pixelated é formado por vários projetos pequenos, que implementam desde a
interface gráfica do \emph{webmail} até a plataforma LEAP que gerencia a busca de
chaves públicas e a criptografia final. Decidimos trabalhar no
\texttt{pixelated-user-agent}, que é o cliente de email, interface direta com o
usuário final.

O \texttt{pixelated-user-agent} roda dentro de uma máquina virtual (para facilitar
uma configuração de alta segurança) e utiliza o \texttt{virtualbox} e o
\texttt{vagrant} para criar esse ambiente virtual de forma automatizada.
O \texttt{vagrant} pode ser visto como um encapsulador e gerenciador de
instâncias virtualizadoras como \texttt{virtualbox}, \texttt{VMware},
\texttt{KVM}, etc.

Existem diversas outras dependências mas essas são automaticamente gerenciadas
dentro da máquina virtual gerada através do \texttt{vagrant}. Portanto nem
usuários e nem
mesmo desenvolvedores precisam se preocupar com elas, a não ser que estejam
trabalhando exatamente na parte da virtualização.

Como decidimos trabalhar no \texttt{webmail}, os conhecimentos necessários
incluíram o \texttt{vagrant} (necessário para subir o ambiente) e conhecimentos
gerais de
desenvolvimento web, como \texttt{HTML}, \texttt{Javascript} e \texttt{CSS}.

Outros conhecimentos úteis são o git, que é usado como controle de versão,
e também o \texttt{IRC}, que é bastante ativo no canal \texttt{\#pixelated} na rede
\texttt{FreeNode}.