\subsection*{\underline{Dependências}:}

O pixelated-user-agent roda dentro de uma máquina virtual (para que haja uma maior segurança).
    Ele possui como dependência o virtualbox e o vagrant, um software que cria e configura ambientes virtuais para desenvolvimento de forma automática. Ele pode ser visto como um encapsulador e gerenciador de instâncias virtualizadoras como VirtualBox, VMware, KVM, etc.
Existem diversas outras dependências também, mas essas são automaticamente gerenciadas dentro da máquina virtual através do vagrant, portanto, usuários e mesmo desenvolvedores não precisam se preocupar com elas.
Quanto aos conhecimentos necessários para trabalhar no Pixelated:
Depende no que for trabalhar. No nosso caso, implementamos uma feature da interface web, portanto era necessario conhecimentos de desenvolvimento web, como HTML, Javascript e CSS.
Para enviar as contribuiçoes é necessario saber usar o git, pois o projeto esta hospedado no Github.

%TODO <Godz, aqui dá pra colocar a descriçao dos repositórios do Pixelated, ou uma versao resumida, aquilo que tá no diario; talvez de pra colocar mais coisas do diario aqui também, voce que sabe>