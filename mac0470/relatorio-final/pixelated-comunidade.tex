\subsection*{\underline{Comunidade}:}

Um ponto que nos chamou a atenção desde o início é a preocupação que os
desenvolvedores do \emph{Pixelated} têm em fomentar o crescimento da comunidade.

Já o primeiro contato foi feito quando, na \emph{CryptoRave}, os desenvolvedores
da
ThoughtWorks deram várias palestras e workshops explicando toda a motivação,
funcionamento técnico e social do projeto e deixando claro o rumo que eles
pretendem que o seu projeto tome: um software livre autônomo que se sustente
pela
própria comunidade, com interferência mínima da própria empresa.

Buscando na internet, foi fácil encontrar informações para o usuário, como por
exemplo
um site bonito e organizado com as explicações do \texttt{porquê} do projeto e
\texttt{como} eles pretendem atingir os objetivos e um demo funcional no próprio
site, onde é possível logar com contas de exemplo e enviar emails
criptografados entre elas.
Além disso, a informação para desenvolvedores também é bem clara, direcionando
para a página no GitHub do projeto, onde cada repositório tem um README detalhado
de como compilá-lo e ainda há um projeto (secure-email) que contém apenas um
grande README expondo todos os problemas em se fazer um sistema de troca de
mensagens seguro, com uma lista extensa de projetos que já implementaram uma ou
outra das tecnologias listadas e sugerindo as soluções mais adequadas para o
próprio Pixelated.

%TODO colocar github da Renata
Além da documentação online, uma das desenvolvedoras do projeto,
\href{https://github.com/re-nobre}{Renata Nobre},
esteve sempre disposta a ajudar o grupo, trocando mensagens instantâneas com um
dos integrantes e até mesmo participando de uma reunião com o CCSL para elaborar
estratégias para a formação da comunidade em torno do projeto.

Outro ponto importante foi a comunicação pelo canal \#pixelated na rede FreeNode
do IRC. Em várias ocasiões lançamos questões no chat e, mesmo com a grande
diferença de fuso horário dos desenvolvedores, que se concentram no Brasil e na
Alemanha, sempre fomos respondidos em um tempo bastante razoável
(de 5 a 30 minutos). Esse foi um canal muito importante na comunicação, evitando
muito tempo parado em problemas de compilação, que foi o nosso maior impedimento
durante o trabalho com o Empathy, por exemplo.

Ao fim do nosso trabalho com a funcionalidade escolhida, não sabíamos exatamente
como proceder. Não havia nenhuma documentação nesse sentido no GitHub, mas nos
sentimos livres para fazer um fork do projeto e enviar um Pull Request. Para
evitar qualquer indelicadeza, enviamos também uma mensagem no IRC e outra para a
Renata, explicando nossa situação de dúvida.

Para nossa surpresa, antes mesmo de sermos respondidos no IRC, recebemos uma
revisão de código de um dos desenvolvedores no GitHub, nos apontando
refatorações e sugestões de melhoria no código que tínhamos enviado há não mais
que 15 minutos.

Concluindo, esse é um projeto bastante atípico (e com certeza muito diferente
do Empathy) pois os desenvolvedores estão bastante ativos na criação de uma
comunidade e fazendo esforços para facilitar ao máximo a vida de quem está
entrando no projeto.
