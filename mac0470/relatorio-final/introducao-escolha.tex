\subsection*{\underline{Escolha do software}:}

No início da disciplina, resolvemos escolher entre um dos projetos
recomendados pelo professor Daniel, assim como fizeram a maioria dos outros
grupos. Em particular demos preferência àqueles documentados no
\href{http://flosscoach.com}{FLOSS Coach},como indicado pelo professor.

Dentre os projetos da lista, levantamos os pontos positivos e negativos
em trabalhar com cada um deles e também ponderamos nossas preferências
pessoais. Finalmente, chegamos a conclusão de trabalhar com o Empathy, um
\emph{messenger} que funciona com diversos protocolos e que faz parte do projeto
Gnome. Consideramos pontos positivos a familiaridade com a linguagem (C++), o
tamanho médio do projeto, a existência de bugs apropriados à nossa capacidade e
o interesse de utilização de um software mensageiro.

Durante o semestre, depois
de muito tempo gasto tentando compilar o código-fonte, perdemos um pouco a
motivação e levantamos a possibilidade de mudar de projeto.
Ao mesmo tempo, tivemos
contato com o projeto Pixelated durante a CryptoRave e decidimos seguir com ele
devido à sua declarada abertura à comunidade, nosso interesse no objetivo de
disseminação de comunicação criptografada e também nas ferramentas utilizadas.

O Pixelated é um projeto que se compromete a servir emails de forma segura e
distribuída com foco na usabilidade do usuário, com a
proposta de substituir os clientes de email atuais por um padrão criptografado.