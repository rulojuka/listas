\subsection*{\underline{Escolha do software}:}
No início da disciplina, resolvemos escolher entre um dos projetos 
recomendados pelo professor Daniel, assim como fizeram a maioria dos outros 
grupos. Em particular demos preferência àqueles documentados no 
\url{flosscoach.com}, como indicado pelo professor. \\
Dentre os projetos da lista, levantamos os pontos positivos e negativos 
em trabalhar com cada um deles e também ponderamos nossas preferências 
pessoais. Finalmente, chegamos a conclusão de trabalhar com o Empathy, um 
messenger que funciona com diversos protocolos e que faz parte do projeto 
Gnome. \\
Consideramos pontos positivos, a familiaridade com a linguagem (C++), o 
tamanho médio de projeto, a existência de bugs apropriados à nossa capacidade e 
o interesse de utilização de um software mensageiro. Durante o semestre, depois 
de muito tempo gasto tentando compilar o projeto, perdemos um pouco a motivação 
e levantamos a possibilidade de mudar de projeto.  Ao mesmo tempo, tivemos 
contato com o projeto Pixelated durante a CryptoRave e decidimos seguir com ele 
devida sua abertura por parte dos membros internos do projeto para contribuição 
por parte de membros externos a este, nosso interesse no objetivo e também nas 
ferramentas utilizadas. \\
Pixelated é um projeto que se compromete a servir de emails de forma segura - 
criptografada - e distribuída com foco na usabilidade do usuário, com a 
proposta de substituir os clientes de email atuais por um novo padrão 
criptografado.

%TODO <Godz, aqui dá pra colocar a descriçao dos repositórios do Pixelated, ou uma versao resumida, aquilo que tá no diario; talvez de pra colocar mais coisas do diario aqui também, voce que sabe>
