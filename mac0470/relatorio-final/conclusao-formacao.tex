\subsection*{\underline{Formação}:}

A partir do que aprendemos na disciplina, concluímos que contribuir para um
projeto de Software Livre é um excelente acréscimo para a formaçao de qualquer
aluno de computação porque são necessárias diversas habilidades para tal, desde
habilidades técnicas até habilidades pessoais de comunicação que, infelizmente,
são constantemente negligenciadas em nossa formação.

Entre as habilidades técnicas desenvolvidas, destaca-se a habilidade de ler
código escrito por outras pessoas, a habilidade de escrever código de qualidade
que seja legível por outros, a utilização de softwares de controle de versão e a
habilidade de aprender novas tecnologias e padrões de projeto. Entre as
habilidades não-técnicas destacamos a comunicação, seja online pelo IRC, seja ao
vivo em conferências de Software e o engajamento em uma comunidade onde todas as
partes têm experiências e conhecimentos diferentes e trabalham para um objetivo
comum. Nosso trabalho com o Pixelated tocou em todos desses pontos.

A experiência de trabalhar em um projeto de software livre desenvolve uma
miríade de habilidades e, portanto, é um grande diferencial na formação do
nosso curso.
