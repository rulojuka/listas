\subsection*{\underline{Tempo gasto}:}

Caio:
16*4 = empathy
Foram quatro dias  inteiros lendo erros e dependências, compilando e ligando tudo. A princípio parece ser simples: compila, instala e ok. Mas na prática não é bem assim. Às vezes as bibliotecas não são encontradas, o binário tem outro nome, as coisas foram para outro lugar e para manter consistências criamos somente links simbólicos, modificamos ou criamos algum .pc (arquivo usado pelo pkg-config).
%TODO TERMINAR ESSA FRASE OU APAGAR Algo que tomou muito tempo foi contornar um bug no momento de compilação <Caio, parece que você parou no meio da escrita dessa frase :P>
8 * 3 = pixelated
4       = PT1
12     = PNT
2       = PT2
12     = relatorio
118h

Leonardo:
Cada palestra exigiu algumas boas horas de preparação, algo de torno de 10 horas. O projeto tambem exigiu varias horas de dedicação em momentos diferentes do semestre. Devo ter gasto umas 4 horas tentando compilar o Empathy no Arch Linux, depois mais umas 4 pesquisando o bug tracker. Com o pixelated foram umas 4 horas também para executá-lo, mais umas 8 horas trabalhando na feature a ser implementada, e mais um pouco para fazer o pull request. No relatório final foram mais umas 10 horas.
Fora isso tambem houve o tempo das aulas, que têm presença obrigatória.

Alessandro:
PT1- 5h
PNT - 25h
PT2 - 5h + pesquisas prévias (~5h)
Empathy: 10h
Pixelated: 20h - Bastante tempo lendo sobre criptografia e os protocolos implementados e em torno de 8h compilando e testando o código.