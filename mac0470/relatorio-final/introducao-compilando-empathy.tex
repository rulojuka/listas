\subsection*{\underline{Início da jornada}:}
Começamos então a compilar o Empathy e tivemos muitos problemas desde o início.
Iniciamos o processo de compilação em 3 distribuições diferentes (ArchLinux,
Debian e Ubuntu) e tivemos muita dificuldade em todas elas.

Como o Empathy faz parte do projeto Gnome, ele possui várias dependências:

\begin{lstlisting}[language=bash]
	gnome-common, gettext, libglib2.0-dev, gtk-doc-tools, libxml2-dev, libtelepathy-glib-dev, libmissioncontrol-client-dev, libtelepathy-farsight-dev, libx11-dev, libgtk2.0-dev, libice-dev{a}, libcanberra-gtk-dev, libgstreamer-plugins-base0.10-dev, libebook1.2-dev, libnotify-dev, libunique-dev, libgnome-keyring-dev, libtelepathy-logger-dev, libwebkitgtk-3.0-dev, libgnutls-dev, libfolks-telepathy-dev, libcanberra-gtk3-dev, libgcr-3-dev, gsettings-desktop-schemas-dev
\end{lstlisting}

Primeira tentativa:
Faltando:
  libyelp-dev yelp-tools -> itstool{a} libxslt1-dev{a} libyelp-dev yelp-tools zip{a}
  libsecret-1
    Velhas:
  {glib-2.0,gio-2.0} >= 2.33.3  -> 2.32.4 encontrado
  telepathy-glib >= 0.22.0 -> 0.18.2 encontrado
       gtk+-3.0 >= 3.5.1 -> 3.4.2 encontrado
glib-2.33.3 -> libffi-dev (3.0.10-3)
gtk+-3.6.0 -> glib-2.0 >= 2.33.1 -> 2.32.4 encontrado
       -> atk >= 2.5.3 -> 2.4.0 encontrado
Segunda tentativa:
O configure passou nos testes para   gerar o primeiro binário mas falhou no segundo:
    Faltando:
gee-0.8
libpulse-mainloop-glib
libpulse
Velhas:
folks >= 0.9.5 -> 0.6.9 encontrado
folks-telepathy >= 0.9.5 -> encontrado 0.6.9
glib-2.0 >= 2.37.6' -> 2.33.3 encontrado
gio-2.0 >= 2.37.6' -> 2.33.3 encontrado
gio-unix-2.0 >= 2.37.6 -> 2.33.3 encontrado
telepathy-glib >= 0.23.2 -> 0.22.0 encontrado
telepathy-logger-0.2 >= 0.8.0 -> 0.4.0 encontrado
gtk+-3.0 >= 3.9.4 -> 3.6.0 encontrado
webkitgtk-3.0 >= 1.9.1 -> 1.8.1 encontrado***

Além das mesmas complicações, agora tivemos que compilar o webkitgtk-3.0.
Esse cara deu muita dor de cabeça. Em particular, neste ponto, descobrimos que deveríamos ter tido instalada uma biblioteca chamada VALA instalada antes de todo o processo de instalação.
Compilamos o vala necessário, recompilamos todas as dependências, e tentamos compilar o webkitgtk-3.0. Durante a compilação, ocorreu um erro dizendo que os arquivos vala estão inconsistentes (redeclaração de classe). A partir desse momento, deixamos o projeto empathy.
É necessário conhecer bem o sistema operacional para tentar compilar o Empathy, isso está comentado na seção 5.4.

Existem alguns projetos do gnome que fazem parte do gnome-love, uma iniciativa que existe para ajudar iniciantes que queiram começar a colaborar com projetos do gnome. Uma das ações desse projeto é a de marcar bugs adequados para iniciantes com a tag gnome-love.
Quando fomos pesquisar bugs do Empathy no bugtracker deles, buscamos por bugs com a tag gnome-love e para nossa surpresa e desapontamento, só encontramos 3. Esse foi mais um dos motivos que nos desmotivou a trabalhar no Empathy.