\subsection*{\underline{Issue}:}
O projeto está no Github, e utiliza o issue tracker do mesmo, buscando lá encontramos a seguinte issue (que na verdade se trata de uma feature request) e trabalhamos nela: https://github.com/pixelated-project/pixelated-user-agent/issues/384
Resumindo, pede-se para implementar float labels (mais informações aqui http://bradfrost.com/blog/post/float-label-pattern) tanto para o Subject quanto para o Body na hora de compor uma mensagem/editar um rascunho.
Apesar de ser uma feature pequena, ela nos exigiu uma boas horas de hacking, pois precisamos entender a estrutura do código do Pixelated para saber aonde fazer as modificações.
    Tivemos que mexer com o HTML, com o CSS e com o Javascript, portanto tivemos que modificar 4 arquivos.
Primeiramente nos preocupamos simplesmente em fazer a feature funcionar, depois, quando conseguimos, tivemos que refatorar o código para ficasse organizado e de boa qualidade, para podermos submetê-lo. Nesse momento tivemos dificuldade com a parte Javascript, pois essa parte do pixelated possui um estrutura bastante complexa.
O processo de submissão foi o seguinte: fizemos um fork do projeto dentro do Github, em seguida clonamos esse fork do repositorio para nossa maquina, criamos uma nova branch para trabalhar, alteramos o código, fizemos o commit, fizemos o push e por ultimo fizemos o pull request, requisitando o merge da nossa branch com o projeto original.
Surpreendentemente, recebemos feedback logo em seguida, em questão de minutos, já nos sugerindo alterações para melhorar a qualidade do código: renomear um selector CSS e mover uma função Javascript que criamos para outro arquivo.
    Não houve tempo suficiente para fazer essas alterações antes da escrita desse relatório, mas faremos assim que possível, para que nossa contribuição seja incluída no Pixelated.