\subsection*{\underline{Pré-requisitos}:}

Para acompanhar as aulas não é necessário muito conhecimento prévio, até porque as palestras apresentadas pelos alunos são introdutórias.
Para contribuir com algum projeto de software livre isso já muda. Porém, é muito difícil responder essa questão, pois o conhecimento necessário é bastante relativo. Depende em qual projeto o aluno vai trabalhar.
Para poder contribuir eficientemente com um projeto de software livre muitas habilidades são necessárias, não só técnicas, como também de comunicação. Quanto mais experiência com diversas linguagens e tecnologias, melhor. Quanto mais experiência com o ecossistema de software livre, melhor. Quanto mais experiência com desenvolvimento de software em equipe, melhor. O difícil é dizer o quanto é necessário de cada um.
É essencial ter bastante vontade de aprender ao cursar essa disciplina, é quase certo que, ao começar a colaborar com um projeto, será necessário aprender algumas tecnologias novas, aprender a se comunicar com aquela comunidade em particular e também aprender como o software funciona a fundo, lendo seu código.
    Vale destacar que conhecer bem o sistema operacional no qual se trabalha ajuda e muito o desenvolvimento do projeto. No caso do empathy por exemplo, pudemos avançar até tal ponto pois tínhamos um conhecimento abrangente do sistema, como ligar bibliotecas, o que significa cada linha de comando que executávamos e como consertar em caso de alguma inconsistência. Para se ter idéia, todas as bibliotecas (dependências) que instalamos, mesmo sendo em uma máquina virtual somente para o projeto da disciplina, foram instaladas em um diretório específico para que tivéssemos certeza de que tudo estava sob controle e que as correções ou features novas pudessem ser lançadas sem problemas. Para isso, foi imprescindível ter um bom conhecimento das ferramentas make, configure e pkg-config, de variáveis de ambiente como $LD_LIBRARY_PATH$, PATH e $PKG_CONFIG_PATH$, e de como funciona o ambiente do terminal (env).
Acreditamos que um aluno que fez apenas as disciplinas básicas pode sim fazer a disciplina, contanto que tenha bastante vontade de aprender, mas provavelmente um aluno mais avançado vai conseguir aproveitá-la melhor.
