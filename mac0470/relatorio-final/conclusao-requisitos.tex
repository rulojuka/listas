\subsection*{\underline{Pré-requisitos}:}

Para acompanhar as aulas não foi necessário muito conhecimento prévio, até
porque as palestras apresentadas pelos alunos foram todas de nível introdutório.

Agora, para contribuir com algum projeto de software livre isso já muda. É muito
difícil responder especificamente essa questão pois o conhecimento necessário
depende muito do projeto no qual o aluno vai trabalhar.

Como já comentamos anteriormente, para poder contribuir eficientemente com um
projeto de software livre muitas habilidades são necessárias, não só técnicas
como também de comunicação. Quanto mais experiência com diversas linguagens e
tecnologias, melhor. Quanto mais experiência com o ecossistema de software
livre, melhor. Quanto mais experiência com desenvolvimento de software em
equipe, melhor. Já o peso de cada habilidade é algo que varia entre projetos.

Para cursar a disciplina é essencial ter bastante vontade de aprender sobre tudo
relacionado a Software Livre, desde seus aspectos legais até os técnicos. Para
a parte prática será necessário aprender algumas tecnologias novas, aprender a
se comunicar com aquela comunidade em particular e também aprender como o
software funciona a fundo, lendo seu código.

Vale destacar que conhecer bem o sistema operacional no qual se trabalha ajuda
muito o desenvolvimento do projeto. No caso do Empathy por exemplo, só pudemos
avançar até tal ponto pois tínhamos um conhecimento abrangente do sistema: como
ligar bibliotecas, o que significa cada linha de comando que executávamos e como
consertar em caso de alguma inconsistência. Por exemplo, todas as
bibliotecas (dependências) que instalamos foram instaladas em um diretório
específico para que tivéssemos certeza de que tudo  estava sob controle e que
as correções ou features novas pudessem ser lançadas sem problemas. Para isso
foi imprescindível ter um bom conhecimento das ferramentas \texttt{make},
\texttt{configure} e \texttt{pkg-config}, de variáveis de ambiente como
\texttt{LD\underline{\space}LIBRARY\underline{\space}PATH}, \texttt{PATH} e
\texttt{PKG\underline{\space}CONFIG\underline{\space}PATH}, e de como funciona
o ambiente do terminal (\texttt{env}).

Acreditamos que um aluno mais avançado no curso se sentirá mais confortável
durante a disciplina porém um aluno dos primeiros anos que tenha
bastante vontade de desenvolver Software Livre também terá plena capacidade de
acompanhar a disciplina.
