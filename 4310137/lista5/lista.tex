\input{../../templates/header.tex}
\providecommand{\e}[1]{\ensuremath{\times 10^{#1}}}
\title{Lista 5 - Física 2}
\author{Alessandro Wagner Palmeira}
\begin{document}
\maketitle

\section{Ex. 1}
O buraco também expande, como se fosse feito pelo material. É fácil perceber
isso imaginando um cubo de moléculas ligadas por molas. Quando aquecemos o material,
o comprimento das molas aumenta e o buraco interno também aumenta.

\section{Ex. 2}


$l = l_{0}(1+\alpha\Delta T)$ \\
$s = l^{2} = (l_{0}(1+\alpha\Delta T))^2 = (l_{0})^2(1+\alpha\Delta T)^2 = s_{0}(1+\alpha\Delta T)^2$ \\

- Magic Happens (Alguma aproximação para $\Delta T$ pequeno?) -

\section{Ex. 3}

Dados: \\
$ \left\{
	  \begin{array}{l l}
		  T_{1} = -78,5ºC \\
		  P_{1}=0,900atm \\
	  \end{array}
  \right.
$ \\
$ \left\{
  	\begin{array}{l l}
	  	T_{2} = 78,0ºC \\
      P_{1}=1,635atm 
	  \end{array}
  \right.
$ \\

Ligando os pontos com uma reta temos que o $x$ com $y=0$ é: \\ \\
$\frac{1,635}{78+78,5+x} = \frac{0,9}{x}$ \\ \\
$(1,635-0,9)x = 0,9 * 156,5$\\ \\
$x = 191,63$\\ \\

Portanto o \textit{zero absoluto} será em \boxed{-78,5 - x = -270,13ºC}\\

\section{Ex. 4}

Dados: \\
$ \left\{
	  \begin{array}{l l}
		  T_{1} = 20ºC \\
		  l_{1}=10cm \\
	  \end{array}
  \right.
$ \\
$ \left\{
  	\begin{array}{l l}
		  T_{2} = 100ºC \\
		  l_{2}=10,015cm \\
	  \end{array}
  \right.
$ \\

Para $\Delta T=80ºC$ temos $\Delta L=0,015cm$\\
$\Delta L = \alpha l_{0}\Delta T$\\
Logo, $\alpha = \frac{0,015cm}{10cm*80ºC} = 1,875\e{-5}ºC^{-1}$\\
$ l = l_{0}(1 + \alpha\Delta T)$\\
Item a)\\
$l = 10(1+1,875\e{-5}*-20)$\\
\boxed{l = 9,99625cm}\\
Item b)\\
$\Delta L = \alpha l_{0}\Delta T$\\
$T_{f} - T_{i} = \Delta T = \frac{\Delta L}{\alpha l_{0}} = \frac{0,009}{1,875\e{-5}*10} = 48ºC$\\
$T_{f} = 48ºC + 20ºC$\\
\boxed{T_{f} = 68ºC}\\
\end{document}
