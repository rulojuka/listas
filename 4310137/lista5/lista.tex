% ***********************************************************
% ******************* PHYSICS HEADER ************************
% ***********************************************************
% Version 2
\documentclass[11pt]{article} 
\usepackage{amsmath} % AMS Math Package
\usepackage{amsthm} % Theorem Formatting
\usepackage{amssymb}	% Math symbols such as \mathbb
\usepackage{graphicx} % Allows for eps images
\usepackage{multicol} % Allows for multiple columns
\usepackage[dvips,letterpaper,margin=0.75in,bottom=0.5in]{geometry}
\usepackage[brazil]{babel}
\usepackage[utf8]{inputenc}
 % Sets margins and page size
\pagestyle{empty} % Removes page numbers
\makeatletter % Need for anything that contains an @ command 
\renewcommand{\maketitle} % Redefine maketitle to conserve space
{ \begingroup \vskip 10pt \begin{center} \large {\bf \@title}
	\vskip 10pt \large \@author \hskip 20pt \@date \end{center}
  \vskip 10pt \endgroup \setcounter{footnote}{0} }
\makeatother % End of region containing @ commands
\renewcommand{\labelenumi}{(\alph{enumi})} % Use letters for enumerate
% \DeclareMathOperator{\Sample}{Sample}
\let\vaccent=\v % rename builtin command \v{} to \vaccent{}
\renewcommand{\v}[1]{\ensuremath{\mathbf{#1}}} % for vectors
\newcommand{\gv}[1]{\ensuremath{\mbox{\boldmath$ #1 $}}} 
% for vectors of Greek letters
\newcommand{\uv}[1]{\ensuremath{\mathbf{\hat{#1}}}} % for unit vector
\newcommand{\abs}[1]{\left| #1 \right|} % for absolute value
\newcommand{\avg}[1]{\left< #1 \right>} % for average
\let\underdot=\d % rename builtin command \d{} to \underdot{}
\renewcommand{\d}[2]{\frac{d #1}{d #2}} % for derivatives
\newcommand{\dd}[2]{\frac{d^2 #1}{d #2^2}} % for double derivatives
\newcommand{\pd}[2]{\frac{\partial #1}{\partial #2}} 
% for partial derivatives
\newcommand{\pdd}[2]{\frac{\partial^2 #1}{\partial #2^2}} 
% for double partial derivatives
\newcommand{\pdc}[3]{\left( \frac{\partial #1}{\partial #2}
 \right)_{#3}} % for thermodynamic partial derivatives
\newcommand{\ket}[1]{\left| #1 \right>} % for Dirac bras
\newcommand{\bra}[1]{\left< #1 \right|} % for Dirac kets
\newcommand{\braket}[2]{\left< #1 \vphantom{#2} \right|
 \left. #2 \vphantom{#1} \right>} % for Dirac brackets
\newcommand{\matrixel}[3]{\left< #1 \vphantom{#2#3} \right|
 #2 \left| #3 \vphantom{#1#2} \right>} % for Dirac matrix elements
\newcommand{\grad}[1]{\gv{\nabla} #1} % for gradient
\let\divsymb=\div % rename builtin command \div to \divsymb
\renewcommand{\div}[1]{\gv{\nabla} \cdot #1} % for divergence
\newcommand{\curl}[1]{\gv{\nabla} \times #1} % for curl
\let\baraccent=\= % rename builtin command \= to \baraccent
\renewcommand{\=}[1]{\stackrel{#1}{=}} % for putting numbers above =
\newtheorem{prop}{Proposition}
\newtheorem{thm}{Theorem}[section]
\newtheorem{lem}[thm]{Lemma}
\theoremstyle{definition}
\newtheorem{dfn}{Definition}
\theoremstyle{remark}
\newtheorem*{rmk}{Remark}

\providecommand{\e}[1]{\ensuremath{\times 10^{#1}}}

% ***********************************************************
% ********************** END HEADER *************************
% ***********************************************************

\providecommand{\e}[1]{\ensuremath{\times 10^{#1}}}
\title{Lista 5 - Física 2}
\author{Alessandro Wagner Palmeira}
\begin{document}
\maketitle

\section{Ex. 1}
O buraco também expande, como se fosse feito pelo material. É fácil perceber
isso imaginando um cubo de moléculas ligadas por molas. Quando aquecemos o material,
o comprimento das molas aumenta e o buraco interno também aumenta.

\section{Ex. 2}


$l = l_{0}(1+\alpha\Delta T)$ \\
$s = l^{2} = (l_{0}(1+\alpha\Delta T))^2 = (l_{0})^2(1+\alpha\Delta T)^2 = s_{0}(1+\alpha\Delta T)^2$ \\

- Magic Happens (Alguma aproximação para $\Delta T$ pequeno?) -

\section{Ex. 3}

Dados: \\
$ \left\{
	  \begin{array}{l l}
		  T_{1} = -78,5ºC \\
		  P_{1}=0,900atm \\
	  \end{array}
  \right.
$ \\
$ \left\{
  	\begin{array}{l l}
	  	T_{2} = 78,0ºC \\
      P_{1}=1,635atm 
	  \end{array}
  \right.
$ \\

Ligando os pontos com uma reta temos que o $x$ com $y=0$ é: \\ \\
$\frac{1,635}{78+78,5+x} = \frac{0,9}{x}$ \\ \\
$(1,635-0,9)x = 0,9 * 156,5$\\ \\
$x = 191,63$\\ \\

Portanto o \textit{zero absoluto} será em \boxed{-78,5 - x = -270,13ºC}\\

\section{Ex. 4}

Dados: \\
$ \left\{
	  \begin{array}{l l}
		  T_{1} = 20ºC \\
		  l_{1}=10cm \\
	  \end{array}
  \right.
$ \\
$ \left\{
  	\begin{array}{l l}
		  T_{2} = 100ºC \\
		  l_{2}=10,015cm \\
	  \end{array}
  \right.
$ \\

Para $\Delta T=80ºC$ temos $\Delta L=0,015cm$\\
$\Delta L = \alpha l_{0}\Delta T$\\
Logo, $\alpha = \frac{0,015cm}{10cm*80ºC} = 1,875\e{-5}ºC^{-1}$\\
$ l = l_{0}(1 + \alpha\Delta T)$\\
Item a)\\
$l = 10(1+1,875\e{-5}*-20)$\\
\boxed{l = 9,99625cm}\\
Item b)\\
$\Delta L = \alpha l_{0}\Delta T$\\
$T_{f} - T_{i} = \Delta T = \frac{\Delta L}{\alpha l_{0}} = \frac{0,009}{1,875\e{-5}*10} = 48ºC$\\
$T_{f} = 48ºC + 20ºC$\\
\boxed{T_{f} = 68ºC}\\
\end{document}
