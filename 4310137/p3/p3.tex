\input{../../templates/header.tex}
\title{P3 - Física 2}
\author{Alessandro Wagner Palmeira}
\begin{document}
\maketitle

\section{Ex. 1}
c)
\begin{itemize}
  \item Lei Zero: ``If C is initially in thermal equilibrium with both A and B, then A and B are also
in thermal equilibrium with each other.'' Sears - University Physics 13th Edition, p. 553

  \item 1ª Lei: \boxed{\Delta U = Q - W}
  \item 2ª Lei:\\
``It is impossible for any system to undergo a process in which it absorbs heat
from a reservoir at a single temperature and converts the heat completely into
mechanical work, with the system ending in the same state in which it began.'' Kelvin-Planck Statement, Sears - University Physics 13th Edition, p. 661 \\
``It is impossible for any process to have as its sole result the transfer of heat from
a cooler to a hotter body.'' Clausius Statement, Sears - University Physics 13th Edition, p. 662
\end{itemize}
d) \\
Eu considerei que a energia cinética era toda convertida em calor. Assim, temos esse calor, mais o calor liberado para esfriar o chumbo de 30º para 0º.
Assim, \boxed{m = 0,22g}, o que eu achei bem pouco, mas faz sentido se pensar que o chumbo esfria fácil (tem um c pequeno) e o gelo tem um calor latente muito alto.

\end{document}
