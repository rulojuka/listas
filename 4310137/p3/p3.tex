\input{../../templates/header.tex}
\title{P3 - Física 2}
\author{Alessandro Wagner Palmeira}
\begin{document}
\maketitle

\section{Ex. 1}
c)
\begin{itemize}
  \item Lei Zero: ``If C is initially in thermal equilibrium with both A and B, then A and B are also
in thermal equilibrium with each other.'' Sears - University Physics 13th Edition, p. 553

  \item 1ª Lei: \boxed{\Delta U = Q - W}
  \item 2ª Lei:\\
``It is impossible for any system to undergo a process in which it absorbs heat
from a reservoir at a single temperature and converts the heat completely into
mechanical work, with the system ending in the same state in which it began.'' Kelvin-Planck Statement, Sears - University Physics 13th Edition, p. 661 \\
``It is impossible for any process to have as its sole result the transfer of heat from
a cooler to a hotter body.'' Clausius Statement, Sears - University Physics 13th Edition, p. 662
\end{itemize}
d) \\
Eu considerei que a energia cinética era toda convertida em calor. Assim, temos esse calor, mais o calor liberado para esfriar o chumbo de 30º para 0º.
Assim, \boxed{m = 0,22g}, o que eu achei bem pouco, mas faz sentido se pensar que o chumbo esfria fácil (tem um c pequeno) e o gelo tem um calor latente muito alto.

\section{Ex. 2}
d)\\
\boxed{\frac{l}{l_{0}} = 1 + \alpha \Delta T} \\
$\frac{l}{l_{0}} = 1,0018 = 1 + \alpha \Delta T$\\
$\alpha \Delta T = 0,0018$\\
$\alpha = \frac{0,0018}{100}$\\
\boxed{\alpha = 1,8\e{-5}}\\
a,b,c)
\boxed{\frac{S}{S_{0}} = 1 + 2\alpha \Delta T}
\boxed{\frac{V}{V_{0}} = 1 + 3\alpha \Delta T}

Então,\\
\boxed{\frac{l}{l_{0}} -1 = 1,8\e{-3} = 0,18\%}
\boxed{\frac{S}{S_{0}} -1 = 3,6\e{-3} = 0,36\%}
\boxed{\frac{V}{V_{0}} -1 = 5,4\e{-3} = 0,54\%}

\section{Ex. 3}
Sears - University Physics 13th Edition, Figure 20.13 - p.664

\section{Ex. 4}
%TODO

\end{document}
