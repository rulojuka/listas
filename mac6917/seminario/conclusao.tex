\begin{frame}
  \frametitle{Melhorias futuras}
  \begin{itemize}
    \item Melhorar o rastreamento com apenas uma performance\pause
    \item Melhorar rastreamento em partes com diferença entre partitura e gravação \pause
    \item Usar mais do que 7 performances como referência \pause
    \item Focar no problema de escolher as melhores performances gravadas para uma certa performance ao vivo \pause
    \item Buscar alternativas para a escolha da mediana
  \end{itemize}
\end{frame}

\begin{frame}
  \frametitle{Dúvidas?}
  \begin{itemize}
    \item Os slides estão disponíveis em \url{https://linux.ime.usp.br/~rulojuka/seminario.pdf}
    \item Acesse \url{https://github.com/compmusMIR} para acessar o código fonte dos slides (Em breve)
  \end{itemize}
\end{frame}

\begin{frame}
  \frametitle{Referências}
  \small
  %Apresentação:\\
  %\vspace{2mm}
  \url{https://www.semanticscholar.org/paper/Real-Time-Music-Tracking-Using-Multiple-Arzt-Widmer/0223b9d27de14f2c158028290782a937ff537786}


\end{frame}
